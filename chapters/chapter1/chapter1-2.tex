\section{Some Preliminaries}

\begin{solution}
  \enum{
  \item
    Suppose for contradiction that $p/q$ is in lowest terms such that $(p/q)^2=3$.
     Then $p^2 = 3q^2$ implying that $p^2$ is a multiple of $3$. Since $3$ is not a perfect square,
     this implies that $p$ is a multiple of $3$. Therefore $p=3r$ for some $r$, substituting $p$ we get,
     $(3r)^2=3q^2$ and $3r^2=q^2$ implying that $q$ is also a multiple of $3$ contradicting the assumption \\
     that $p/q$ is already at its lowest terms.

     The similar argument works for $\sqrt{6}$.

  \item
    $\sqrt{4}$ is a perfect square.
    This is where the proof for Theorem 1.1.1 breaks down.
  }
\end{solution}

\begin{solution}
  If $r=0$ then $2^{r} = 1 \neq 3$. If $r \neq 0$ then set $p/q = r$ to get $2^{p} = 3^{q}$ which is not possible since
  $2$ and $3$ share no factors.
\end{solution}


\begin{solution}
  \enum{
  \item
    It does not hold for $A_{n}=\{m \mid m \geq n: \forall n \in \mathbb{N}\}$

  \item
    It does not hold for $A_{n}=\{m \mid n \leq m \leq n + 100 \forall n \in \mathbb{R}\}$.

    \textbf{Update:}
    The initial reasoning is wrong because in that $A_{n} \nsupseteq A_{m} for n \leq m$.
    It is true because, eventually there must exist $A_{m}$ for some $m$ and the pattern continues
    $A_{1},\ldots,A_{m-1}, A_{m}, A_{m}, A_{m}, \ldots, A_{m}$.
  
  \item
    $x \in A \cap (B  \cup  C)$ means $x \in A$ and $x \in B$ or $x \in C$.
    If $x \in C$ then $x \in (A \cap B) \cup C$.
    If $x \in B$ then $x \in A \cap B$, which implies $x \in (A \cap B) \cup C$.
    This shows that $(A \cap B) \cup C \subseteq A \cap (B \cup C)$.

    Again,
    $x \in (A \cap B) \cup C$ means $x \in$ A and $x \in B$ or $x \in C$.

    If $x \in C$ then, $x \in (A \cap B) \cup C$.
    If $x \in A$ and $x \in B$ then, $x \in A \cap (B \cup C)$.
    This shows that $A \cap (B \cup C) \subseteq (A \cap B) \cup C$.

    Thus, $A \cap (B \cup C) = (A \cap B) \cup C$. 

  \item
    Similar as above

  \item
    $x \in A \cap (B \cup C)$ means $x \in A$ and $x \in B$ or $x \in C$.
    This means $x \in A \cap B$ or $x \in A \cap C$.
    So, $(A \cap B) \cup (A \cap C) \subseteq A \cap (B \cup C)$

    $x \in (A \cap B) \cup (A \cap C)$ means $x \in A \cap B$ or $x \in A \cap C$.
    This means, $x \in A$ and $x \in B$ or $x \in C$.
    So, $A \cap (B \cup C) \subseteq (A \cap B) \cup (A \cap C)$.

    Hence, $A \cap (B \cup C) = (A \cap B) \cup (A \cap C)$
  }
\end{solution}

\begin{solution}
  $$
  \begin{array}{lccccc}
    1 & 3 & 6 & 10 & 15 & \cdots \\
    2 & 5 & 9 & 14 & \cdots & \\
    4 & 8 & 13 & \cdots & & \\
    7 & 12 & \cdots & & & \\
    11 & \ldots & & & & \\
    \vdots & & & & &
  \end{array}
  $$

  $A_{i}$ is the $i^{th}$ row.
\end{solution}

\begin{solution}
  \enum{
  \item
    $x \in (A \cap B)^{c}$ means $x \notin A$ or $x \notin B$.
    This implies, $(A \cap B)^{c} \subseteq A^{c} \cup B^{c}$

  \item
    $x \in A^{c} \cup B^{c}$ means $x \in A^{c}$ or $x \in B^{c}$.
    This means, $x \notin A \cap B$, which implies $x \in (A \cap B)^{c}$.

    $(A \cap B)^{c} \supseteq A^{c} \cup B^{c}$

  \item
    $(A \cap B)^{c} \subseteq A^{c} \cup B^{c}$ and  $(A \cap B)^{c} \supseteq A^{c} \cup B^{c}$.
    So  $(A \cap B)^{c} = A^{c} \cup B^{c}$
    
  }

\end{solution}

\begin{solution}
  \enum{
  \item
    $\abs{a + b} \leq \abs{a} + \abs{b}$.
    Squaring, we get,
    $(a + b)^{2} \leq (\abs{a} + \abs{b})^{2}$
    $a^{2} + 2ab + b^{2} \leq \abs{a}^{2} + 2\abs{a}\abs{b} + \abs{b}^{2}$
    $a^{2} + 2ab + b^{2} \leq a^{2} + 2\abs{a}\abs{b} + b^{2}$.
    For $a,b > 0$ or $a,b < 0$, $ab = \abs{a}\abs{b}$.
    Therefore,  $a^{2} + 2ab + b^{2} = a^{2} + 2\abs{a}\abs{b} + b^{2}$.
    So the triangle ineqaulity holds.

  \item
    The inequality reduces to,
    $ab \leq \abs{a}\abs{b}$
    which is true when $ab < 0$ as $\abs{a}\abs{b}$ is always greater than 0.
    since squaring preserves inequality this implies that $\abs{a+b} \leq \abs{a} + \abs{b}$.

  \item
    $\abs{a-b} = \abs{a - c + c - b} \leq \abs{a - c} + \abs{c - b}$,
    $abs{a - c} + \abs{c - b} = \abs{a - c} + \abs{c - d + d - b} \leq \abs{a - c} + \abs{c - d} + \abs{d - b}$.

  \item
    Then inequality reduces to,
    $-\abs{a}\abs{b} \leq -ab$ which is true.
    Alternatively, we can also prove it using the triangle inequality and $a = a - b + b$.
    We know $\abs{\abs{a} - \abs{b}} = \abs{\abs{b} - \abs{a}}$, so we can assume $\abs{a} > \abs{b}$ without the loss of generality.
    Then $\abs{\abs{a} - \abs{b}} = \abs{a} - \abs{b} = \abs{(a - b) + b} - \abs{b} \leq \abs{a - b} + \abs{b} - \abs{b} = \abs{a - b}$.
  }
\end{solution}

\begin{solution}
  \enum{
  \item
    $f(A) = [0, 4]$ and $f(B) = [1, 16]$.
    In this case,
    $A \cap B = [1,2]$, $f(A) \cap f(B) = [1,4] = f(A \cap B)$.
    $A \cup B = [0, 4]$, $f(A) \cup f(B) = [0, 16] = f(A \cup B)$.

  \item
    For $A = [-b, -a]$ and $B = [a, b]$, $A \cap B = \emptyset$ which implies $f(A \cap B) = \emptyset$.
    But $f(A) = [a^{2}, b^{2}]$ and $f(B) = [a^{2}, b^{2}]$ with $f(A) \cap f(B) = [a^{2}, b^{2}] \neq \emptyset = f(A \cap B)$.


  \item
    $y \in g(A \cap B)$ means $y = g(x)$ for some $x \in A \cap B$.
    This implies $x \in A$ and $x \in B$ which means,
    $y \in g(A)$ and $y \in g(B)$.
    So, $g(A \cap B) \subseteq g(A) \cap g(B)$.

    The reverse inclusion does not apply when $g$ is not 1-1 function because it is possible to get $y = g(x) = g(x')$ where
    $x \neq x'$

  \item 
    \textbf{Conjecture: $g(A \cup B) = g(A) \cup g(B)$}
    
    $y \in g(A \cup B)$ means $\exists{x} \in A \cup B$ such that $y = g(x)$.
    This means, $x \in A$ or $x \in B$, hence $y \in g(A) \cup g(B)$.
    So, $g(A \cup B) \subseteq g(A) \cup g(B)$.

    Now, $y \in g(A) \cup g(B)$ means  $\exists{x} \in A$ or  $\exists{x'} \in B$ such that
    $y = g(x) = g(x')$. Regardless of the equality of $x$ and $x'$, $x,x' \in A \cup B$ meaning,
    $g(A \cup B) \supseteq g(A) \cup g(B)$.

    Hence, $g(A \cup B) = g(A) \cup g(B)$
  }
\end{solution}


\begin{solution}
  \enum{
  \item
    $f(x) = 2n$
  \item
    $f(1) = 1$ and $f(n) = n - 1$
  \item
    $f(n) = n / 2$ when $n$ is even and $f(n) = -(n + 1)/2$ when $n$ is even.
  }
\end{solution}

\begin{solution}
  \enum{
  \item
    $A = [0, 4]$
    $B = [-1, 1]$
    $f^{-1}(A) = [-2,2]$
    $f^{-1}(B) = [-1, 1]$

    $A \cap B = [0, 1]$
    $f^{-1}(A \cap B) = [-1, 1] = f^{-1}(A) \cap f^{-1}(B)$

    $A \cup B = [-1, 4]$
    $f^{-1}(A \cup B) = [-2, 2] = f^{-1}(A) \cup f^{-1}(B)$

  \item
    $x \in g^{-1}(A \cap B)$ means that $g(x) \in A \cap B$.
    which means $g(x) \in A$ and $g(x) \in B$, this implies
    $x \in g^{-1}(A)$ and $x \in g^{-1}(B)$.
    So, $ g^{-1}(A \cap B) \subseteq  g^{-1}(A) \cap  g^{-1}(B)$
    Now,
    $x \in g^{-1}(A) \cap  g^{-1}(B)$ means that $x \in g^{-1}(A)$ and $x \in  g^{-1}(B)$.
    Which is same as $g(x) \in A$ and $g(x) \in B$.
    Then we get, $g(x) \in A \cap B$ this implies that $x \in g^{-1}(A \cap B)$.
    Therefore,  $g^{-1}(A) \cap  g^{-1}(B) \subseteq g^{-1}(A \cap B)$.

    Hence, $g^{-1}(A) \cap  g^{-1}(B) = g^{-1}(A \cap B)$.

    Same logic for the union case. Alternatively, we can also use the fact we proved in 1.2.7 (d).
  }
\end{solution}


\begin{solution}
  enum{
  \item
    The reverse inclusion does not hold for $a = b$.
  \item
    Same reason as above.
  \item
    $b < b + \epsilon$ for every $\epsilon > 0$, this implies,
    $a < b + \epsilon$. 

    For backward inclusion, $a < b + \epsilon$ for every $\epsilon > 0$.
    $b < a$ is not possible because we will reach a contradiction for $\epsilon_{0} = a - b$.
    So either $a < b$ or $a = b$.
  }
\end{solution}


\begin{solution}
  \enum{
  \item
    The claim is true.
    Negation: For all real numbers satisfying $a > b$, $a + 1/n \geq b$ for all $n \in \mathbb{N}$.

  \item
    The claim is false.
    Negation: For all real numbers $x > 0$, there exists $n \in \mathbb{N}$ such that $x \geq n$.

  \item
    The claim is true.
    Negation: There exists two distinct real numbers $a$ and $b$ such that $a < b$ with $r < a$ and $r > b$ for all $r \in \mathbb{Q}$.
    
  }
\end{solution}

\begin{solution}
  \enum{
  \item
    For $n = 1$, $y_{1} = 6 > -6$. It is true for the base case.
    Suppose inductively that $y_{n} > -6$.
    Now, $y_{n + 1} = (2y_{n} - 6) / 3$ we get,
    $y_{n + 1} = 2/3 y_{n} - 2$
    We know, $2/3 y_{n} > -4$, from this we get,
    $y_{n+1} = 2/3 y_{n} - 2 > -6$.

  \item
    For $n = 1$, $y_{2} = 2 < 6 = y_{1}$. 
    Suppose inductively that $y_{n+1} < y_{n}$. Then we get,
    $2y_{n + 1} < 2y_{n}$,$2y_{n + 1} - 6 < 2y_{n} - 6$, ,$(2y_{n + 1} - 6) / 3 < (2y_{n} - 6) / 3$, $y_{n+2} < y_{n + 1}$.
  }
\end{solution}


\begin{solution}
  \enum{
  \item
    We know, $(A_{1} \cup A_{2})^{c} = A_{1}^{c} \cap A_{2}^{c}$.
    So it is true for the base case.
    Suppose for induction that,
    $(A_{1} \cup A_{2} \cup \cdots \cup A_{n})^{c} = A_{1}^{c} \cap A_{2}^{c} \cap \cdots A_{n}^{c}$.

    Then,
    $(A_{1} \cup A_{2} \cup \cdots \cup A_{n+1})^{c} = (A_{1} \cup A_{2} \cup \cdots \cup A_{n} \cup A_{n + 1})^{c}$

    $(A_{1} \cup A_{2} \cup \cdots \cup A_{n} \cup A_{n + 1})^{c} = (\bigcup_{i=1}^{n} A_{i} \cup A_{n+1})^{c}$

    $(\bigcup_{i=1}^{n} A_{i} \cup A_{n+1})^{c} = (\bigcup_{i=1}^{n} A_{i})^{c} \cap A_{n+1}^{c}$

    $(\bigcup_{i=1}^{n} A_{i})^{c} \cap A_{n+1}^{c} = \bigcap_{i=1}^{n} A_{i}^{c} \cap A_{n+1}^{c} = \bigcap_{i=1}^{n + 1} A_{i}^{c}$.
  \item
    $B_{n} = \{m \in \mathbb{N} : m \geq n \}$
  \item
    $x \in (\bigcup_{i=1}^{n} A_{i})^{c}$ means that $x \notin A_{i}$ so $x \in A_{i}^{c}$ for every $i$.
    Then, $x \in \bigcap_{i=1}^{n} A_{i}^{c}$
    So, $(\bigcup_{i=1}^{n} A_{i})^{c} \subseteq \bigcap_{i=1}^{n} A_{i}^{c}$


    Now, $x \in \bigcap_{i=1}^{n} A_{i}^{c}$ means that $x \in A_{i}^{c}$ so $x \notin A_{i}$ for every $i$.
    Then, $x \notin \bigcup_{i=1}^{n} A_{i}$ which implies $x \in (\bigcup_{i=1}^{n} A_{i})^{c}$.
    So, $(\bigcup_{i=1}^{n} A_{i})^{c} \supseteq \bigcap_{i=1}^{n} A_{i}^{c}$.


    Thus, $(\bigcup_{i=1}^{n} A_{i})^{c} = \bigcap_{i=1}^{n} A_{i}^{c}$
  }
\end{solution}

