\section{Cardinality}

\begin{solution}
  $g(i) = f(n_{i})$ where $n_{i} = \min\{n \in \mathbb{N} - [n_{1}, n_{i - 1}] : f(n_{i}) \in A\}$.
  $g: \mathbb{N} \rightarrow A$ is 1-1 and onto. Thus $A$ is countable. 
\end{solution}

\begin{solution}

  The flaw is that the nested interval theorem only applies to set that has the least upper bound property.
  $\mathbb{Q}$ does not have this property.
\end{solution}

\begin{solution}
  \enum{
  \item
    $B_{2} \subseteq A_{2}$ thus $B_{2}$ is countable.
    Define,
    $g(1) = f_{1}(1)$
    $g(k) = f_{1}(k - 1) if k is odd$
    $g(k) = f_{2}(k - 1) if k is even$

    $g$ is 1-1 and onto. Thus, $A_{1} \cup B_{2}$ is countable.


    Suppose for induction it is true for $n$.
    Now $A_{1} \cup \cdots \cup B_{n} \cup B_{n + 1}$ is union of countable sets, which by induction hypothesis is countable.
    
  \item
    Induction does not apply for the infinite case. It only applies for $n^{th}$ case.
    
  \item
    The rearrangement of $\mathbb{N}$ gives us the disjoint sets $C_{n}$ such that $\bigcup_{n = 1}^{\infty}C_{n}=\mathbb{N}$.

    Let $B_{n}$ be disjoint sets constructed as $B_{1} = A_{1}, B_{2} = A_{1} \ B_{1}, \ldots$

    we want to do something like:

    $$f(\mathbf{N}) = f\left(\bigcup_{n=1}^\infty C_n\right) = \bigcup_{n=1}^\infty f_n(C_n) = \bigcup_{n=1}^\infty B_n = \bigcup_{n=1}^\infty A_n$$
    $\exists$ bijection $f_{n}: C_{n} \to B_{n}$ because $B_{n}$ is countable.
     $f : \mathbb{N} \to \bigcup_{n=1}^\infty B_n$ as

     $$
     f(n) = \begin{cases}
              f_{1}(n) &\text{if } n \in C_{1} \\
              f_{2}(n) &\text{if } n \in C_{2} \\
              \vdots
            \end{cases}
     $$

     \enumr{
     \item Since each $C_{n}$ is disjoint and each $f_{n}$ is 1-1,
       $f(n_{1}) = f(n_{2})$ implies $n_{1} = n_{2}$ meaning $f$ is 1-1.
       
     \item Since any $b \in \bigcup_{n=1}^{\infty} B_{n}$ has $b \in B_{n}$ for some $n$, we know $b = f_{n}(x)$ has a solution since $f_{n}$ is onto.
       Letting $x = f_{n}^{-1}(b)$ we have $f(x) = f_{n}(x) = b$ since $f_{n}^{-1}(b) \in C$ meaning $f$ is onto.
     }

     By (i) and (ii) $f$ is bijective and so $\bigcup_{n=1}^{\infty} B_{n}$ is countanble.  And since
      $$\bigcup_{n=1}^\infty B_n = \bigcup_{n=1}^\infty A_n$$
      We have that $\bigcup_{n=1}^\infty A_n$ is countable, completing the proof.
   }
\end{solution}

\begin{solution}
  \enum{
  \item We will start by finding $f : (-1, 1) \to \mathbf{R}$ and then transform it to $(a, b)$. Example 1.5.4 gives a suitable $f$
    $$
    f(x) = \frac{x}{x^2 - 1}
    $$
    The book says to use calculus to show $f$ is bijective, first we will examine the derivative
    $$
    f'(x) = \frac{x^2 - 1 - 2x^2}{(x^2 - 1)^2} = - \frac{x^2 + 1}{(x^2 - 1)^2}
    $$
    The denominator and numerator are positive, so $f'(x) < 0$ for all $x \in (0,1)$.
    This means no two inputs will be mapped to the same output, meaning $f$ is one to one (a rigorous proof is beyond our current ability)

    To show that $f$ is onto, we examine the limits
    $$
    \begin{aligned}
      \lim_{x \to 1^{-}} \frac{x}{x^2 - 1} &= - \infty \\
      \lim_{x \to {-1}^{+}} \frac{x}{x^2 - 1} &= + \infty
    \end{aligned}
    $$
    Then use the intermediate value theorem to conclude $f$ is onto.

    Now we shift $f$ to the interval $(a, b)$
    $$
    g(x) = f\left(\frac{2x - 1}{b - a} - a\right)
    $$
    Proving $g(x)$ is also bijective is a straightforward application of the chain rule.
  \item We want a bijective $h(x)$ such that $h(x) : (a, \infty) \to (-1, 1)$ because then we could compose them to get a new bijective function $f(h(x)) : (a, \infty) \to \mathbf{R}$.

    Let
    $$
    h(x) = \frac{2}{x - a + 1} - 1
    $$
    We have $h : (a, \infty) \to (1, -1)$ since $h(a) = 1$ and $\lim_{x \to \infty} h(x) = 1$.

    Meaning that $f(h(x)) : (a, \infty) \to \mathbf{R}$ is our bijective map.
  \item With countable sets adding a single element doesn't change cardinality since we can just shift by one to get a bijective map. we'll use a similar technique here to essentially outrun our problems. Define $f : [0,1) \to (0,1)$ as
    $$
    f(x) = \begin{cases}
      1/2 &\text{if } {x=0} \\
      1/4 &\text{if } {x=1/2} \\
      1/8 &\text{if } {x=1/4} \\
      \vdots \\
      x &\text{otherwise}
    \end{cases}
    $$
    Now we prove $f$ is bijective by showing $y = f(x)$ has exactly one solution for all $y \in (0,1)$.

    If $y = 1/2^n$ then the only solution is $y = f(1/2^{n-1})$ (or $x=0$ in the special case $n=1$),
    If $y \ne 1/2^n$ then the only solution is $y = f(y)$.
  }
\end{solution}

\begin{solution}
  \enum{
  \item Because there is 1-1 and onto mapping from $A$ to $A$. Simply, $f(x) = x$
    
  \item If $A ~ B$ then, $f: A \to B$ is 1-1 and onto, so define, $g: B \to A$ as $g = f^{-1}$, $g$ is also 1-1 and onto.
    
  \item If $A ~ B$ then, $f: A \to B$ is bijection, and $B ~ C$ then $g: B \to C$ is a bijection.
    Thus $h: A \to C$ with $h = g(f(a))$ is a bijection. Hence $A ~ C$
  }
\end{solution}

\begin{solution}
  \enum{
  \item $A_{n} = (n, n + 1), n \in \mathbb{N}$.
    
  \item
    No such collection exists.
    If the intervals contain two irrational numbers, $\alpha, \beta$ then there exists an irrational number in between.
    So the intervals are not disjoint.

    
    If the interval $I$ is defined as $(\alpha, \beta)$ where beta is the smallest irrational number greater than $\alpha$.
    Then, $I \subseteq \mathbb{Q}$ thus $I$ is countable.
  }
\end{solution}

\begin{solution}
  \enum{
  \item $f(x) = 2x$.
  \item
    Let $g: S \to (0, 1)$ be a function that interleaves decimals in the representation without trialing nines, padding with zeros if necessary.
    $g(0.32, 0.45) = 0.3245, g(0.1\bar 9, 0.2) = g(0.2, 0.2) = (0.22), g(0.1, 0.23)=g(0.1203), g(0.1, 0.\bar 2) = 0.12\bar 02$ etc.

    Every real number can be written with two digits representations, one with trailing 9's and one without.
    $g(x, y) = 0.d_{1}d_{2}\ldots\bar 9$ is impossible since it would imply
    $x = 0.d_{1}\ldots\bar 9$ and $y = 0.d_{1}\ldots\bar 9$ but the definition of $g$ does not allow this,
    therefore $g(s)$ is unique and so $g$ is 1-1.

    Is $g$ onto? No since $g(x,y) = 0.1$ has no solutions, since we would want $x = 0.1$ and $y = 0$ but $0 \notin (0, 1)$.
  }
\end{solution}

\begin{solution}
   Notice $B \cap (a, 2)$ is finite for all $a > 0$, since if it was infinite we could make a set with sum greater then two.
  And since $B$ is the countable union of finite sets $\bigcup_{n=1}^\infty B \cap (1/n, 2)$, $B$ must be countable or finite.
\end{solution}


\begin{solution}
  \enum{
  \item $x^{2} = 2$, $x^{3} = 2$, $x^{4} - 10x^{2} + 1 = 0$
    
  \item Basically, $A ~ \mathbb{Z}^{n} ~ \mathbb{N}^{n} ~ \mathbb{N}$
    
  \item $A = \bigcup_{n = 1}^{\infty} A_{n}$ is countable because $A_{n}$ is countable.
    
  \item Since $\mathbb{R} = A \cup T$ where $T$ is the set of transcendental numbers. Since $\mathbb{R}$ is uncountable and $A$ is countable, it must be that $T$ is uncountable.
  }
\end{solution}

\begin{solution}
  \enum{
  \item
    Suppose $C \cap [a, 1]$ is countable $\forall a \in (0, 1)$.
    Then,

    $$
    \bigcup_{n = 1}^{\infty} C \cap [1/n , 1] = C \cap [0, 1] = C
    $$

    This implies that $C$ is countable. Contradiction.
    
  \item
    For $epsilon > 0$, $C \cap [\alpha + \epsilon, 1]$ is countable, otherwise it will be in $A$ contradicting $\alpha = \sup A$.

    Now,
    $$
    \bigcup_{n = 1}^{\infty} C \cap [\alpha + 1/n , 1] = C \cap [\alpha, 1]
    $$
    Thus $ C \cap [\alpha, 1]$ being union of countable sets is countable.
    
  \item No, consider the set $C = \{ 1/n : n \in \mathbb{N}\}$ it has $C \cap [a, 1]$ finite for every $a$ but
    $C \cap [0, 1]$ is infinite.
    
  }
\end{solution}

\begin{solution}
  \enum{
  \item $$ h(x) = \begin{cases}
             f(x) & \text{if} {x \in A} \\
             g^{-1}(x) & \text{if} {x \in A'}
           \end{cases}           
           $$

      
           
         \item
           If $A_{1} = \emptyset$ then $g$ is already a bijection.
           Assume $A_{1} \neq \emptyset$. Then,
           $A_{2} = g(f(A_{1}))$, which is by definition of $A_{1}$ disjoint.

           Suppose inductively that $A_{1} \cap A_{n} = \emptyset$.
           Then, $A_{1} \cap A_{n + 1} = A_{1} \cap g(f(A_{n}))$

           $f^{-1}(g^{-1}(A_{1})) \cap A_{n} = \emptyset$.

           Now, for every $j > k$, $A_{j} \cap A_{k} = \emptyset$.

           $A_{j} \cap A_{k} = h(A_{j}) \cap h(A_{k})$
           we get,
           $h^{k}(A_{j -k}) \cap h^{k}(A_{1}) = h^{k}(A_{j - k} \cap A_{1})$
           which is,
           $h^{k}(\emptyset) = \emptyset$.

           Since $f$ is 1-1.
           $f(A_{j}) \cap f(A_{k}) = f(A_{j} \cap A_{k}) = f(\emptyset) = \emptyset$
         \item
           $f(\bigcup_{n=1}^{\infty}A_{n} = \bigcup_{n=1}^{\infty}f(A_{n})$
           
         \item
           Now every $x \notin A$ implies $x \in g(Y)$ thus $x \in A'$.
           If $y \in B \mid g(y) = x \in A'$, then this implies $y = g(f(A_{n}))$ for some $n$ and hence $x \in A_{n + 1} \subseteq A$. Contradiction.
           Thus $y \in B'$.
  }
\end{solution}
