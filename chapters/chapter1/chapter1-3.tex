\section{The Axiom of Completeness}

\begin{solution}
  \enum{
  \item
    For a non-empty set $A$, $i = inf A$ is the infimum of $A$ if and only if
    \enumr{
    \item $i \leq a,  \forall a \in A$
    \item If $b$ is a lower bound of $A$ then $b \leq i$.
    }

  \item
    Assume $i \in \mathbb{R}$ is a lower bound for a set $A \subseteq \mathbb{R}$. Then,
    $i = inf A$ if and only if, for every choice of $\epsilon > 0$ there exists an element $a \in A$ satisfying
    $i + \epsilon > a$.


    \begin{proof}
      $(\implies)$ Suppose for contradiction that $i + \epsilon \leq a$ for all $\epsilon > 0$.
      This means that $i + \epsilon$ is a lower bound of $A$.
      But $i =  infA$ and $i < i + \epsilon$. Contradiction. Thus, $i + \epsilon > a$ for some $a \in A$.

      $(\impliedby)$ We have some $a \in A$ such that $i + \epsilon > a$ for all $\epsilon > 0$.
      Then we get $\epsilon > a - i$ this implies $a > i$ for every choice of $\epsilon > 0$.
      So $i$ is a lower bound of $A$.
      Suppose $b > i$ and $b = inf A$, then for $\epsilon_{0} = b - i > 0$.
      $i + \epsilon_{0} > a$.
      $i + b - i > a$
      $b > a$. Contradiction, since $b = inf A$, $b \leq a \forall a \in A$.
      So $b \neq inf A$. Thus, $i = inf A$.
    \end{proof}
  }
\end{solution}

\begin{solution}
  \enum{
  \item $B = \{x \}$. $\inf B = x = \sup B$. Thus $\inf B \geq \sup B$.
  \item Impossible.
  \item $B = \{x \in \mathbb{Q} \mid 2 < x \leq 4 \}$. $sup B = 4$ and $inf B = 2 \notin B$
  }
\end{solution}

\begin{solution}
  \enum{
  \item
    $\sup B$ is a lower bound for $A$.
    Let $\beta$ be a lower bound of $A$ such that $\beta > \sup B$.
    Since $\beta$ is a lower bound of $A$ so $\beta \in B$ by the definition of B.
    This means $\beta \leq \sup B$ by definition of supremum.
    Contradiction.
    So such $\beta$ does not exist.
    Therefore $\sup B = \inf A$.
  \item For any non-empty set $A$, its infimum is the supremum of the set containing all of its lower bounds.
    Defining supremum as axiom is enough to derive the existence of infimum.
  }
\end{solution}

\begin{solution}
  \enum{
  \item
    $\sup (A_{1} \cup A_{2}) = \sup \{ \sup A_{1}, \sup A_{2}\}$.
    $\sup(\bigcup_{k = 1}^{n} A_{k}) = \sup \{ \sup A_{k} \mid k = 1, 2, \cdots , n \}$
  \item Not necessarily, because $\bigcup_{k = 1}^{\infty} A_{k}$ can be unbounded.
  }
\end{solution}

\begin{solution}
  \enum{
  \item Given $c \geq 0$, $a \leq \sup A, \forall a \in A$, then we have $ca \leq c\sup A$.
    This means $c \sup A$ is an upper bound for the set $cA$.
    Let $\alpha$ be an upper bound of $cA$ such that $\alpha < c\sup A$.
    Since $ca \leq \alpha$, this means $a \leq \alpha / c$.
    This implies, $\alpha / c > \sup A$, we get, $ \alpha > c\sup A$. Contradiction.
    So $\sup(cA) = c \sup A$
  \item When $c < 0$, $\inf(cA) = c \sup A$.
  }
\end{solution}

\begin{solution}
  \enum{
  \item
    $a \leq s$ and $b \leq t$. Adding these two inequalities we get,
    $a + b \leq s + t$, this means that $s + t$ is an upper bound for $A + B$.
  \item $a + b \leq u$ then $b \leq u - a$. Now, $\sup B = t \leq u - a$.
  \item $a \leq u - t$, so $s \leq u - t$, we get, $s + t \leq u$.
    Thus $\sup(A + B) = s + t$
  \item Suppose for contradiction that $s + t - u > 0$, then
    $s - (s + t - u) < a$ we get,
    $-t + u < a$, $t > u - a$. Contradiction. So $s + t - u$ must be less than or equal to $0$.
    So, $s + t - u \leq 0$ this implies $s + t \leq u$. Thus, $\sup(A + B) = s + t$.
  }
\end{solution}


\begin{solution}
  Since $a$ is an upper bound it must be that $\sup A \leq a$
  If $\sup A < a$ then it does not work for $a$ because $a \in A$ and $a > \sup A$.
  Thus, $\sup A = a$.
\end{solution}

\begin{solution}
  \enum{
  \item supremum is $1$ and infimum is $0$
  \item sup is $1$ and inf is $-1$
  \item inf is $1/4$ and sup is $1/3$
  \item inf is $0$ and sup is $1$
  }
\end{solution}

\begin{solution}
  \enum{
  \item
    For every $b \in B$, suppose for contradiction that $b < \sup A$.
    Then, this means $\sup A < \sup B$ is an upper bound for $B$. Contradiction.
    So $\exists b \in B$ such that $b > a, \forall a \in A$.
  \item $A = \{x \mid x \leq 1\}$, $B = \{ x \mid x < 1 \}$.
  }
\end{solution}

\begin{solution}
  \enum{
  \item
    The definition readily implies that $\sup A \leq \inf B$. Now suppose for contradiction that $\exists x \in \mathbb{R}$ such that
    $\sup A < x < \inf B$. Then this means $x \notin A \cup B$. So, $A \cup B \neq \mathbb{R}$. Contradiction. So such a $x$ does not exist.
    This means $\sup A = \inf B$
    Now let $c = sup A$ then $\forall x \in A$, $x \leq c$ and $\forall x \in B$, $x \geq c$.
    
  \item Let $B = \{ b \mid b > x, \forall x \in E\}$.
    $B$ is not empty because $E$ is bounded above so there exists atleast one $b \in B$.
    Since $E$ is only bounded above $E \subset \mathbb{R}$ and $B = E^{c} \subset \mathbb{R}$.
    Thus $E \cup B = \mathbb{R}$
    By the cut property $\exists c$ such that $c \geq x, \forall x \in E$  and $c \leq x, \forall x \in B$.
    This means,
    \enumr{
    \item $c$ is an upper bound of A
    \item $c \leq b$ means $c$ is the least upper bound.
    }

    $c = \sup E$. This proves existence of $\sup E$.
    
  \item $A = \{ x \mid x^{2} < 2\} \cup [-\infty, 0]$ and $B = \{x \mid x^{2} > 2 \} - [-\infty, 0]$.
    $A \cup B = \mathbb{Q}$ but $\nexists c \in \mathbb{Q}$ such that $x \leq c, \forall x \in A$ and $c \leq x, \forall x \in B$.
  }
\end{solution}


\begin{solution}
  \enum{
  \item
    True.
    Suppose for contradiction $\sup A > \sup B$, then $\forall b \in B, b \leq \sup B$.
    Also since $A \subseteq B$ then $\forall a \in A, a \leq \sup B$ with $\sup A > \sup B$. Contradiction.
    Thus $\sup A \leq \sup B$.
  \item True.
    $c = (\sup A + \inf B) / 2$.
  \item False
    $A = \{x \mid x < 1\}$ and $B = \{x \mid x > 1\}$.
    Then, $\sup A = 1 = \inf B$.
  }
\end{solution}
